% !TeX root = ./main.tex

\stusetup{
  title              = {汕头大学\\学位论文模板示例文档 \stuthesisversion},
  title*             = {An example of thesis template for Shantou University \stuthesisversion},
  author             = {李泽平},
  author*            = {Li Zeping},
  speciality         = {计算机科学与技术},
  speciality*        = {Computer Science and Technology},
  supervisor         = {XXX~教授},
  supervisor*        = {Prof. XXX},
  % date               = {2017-05-01},  % 默认为今日
  % professional-type  = {专业学位类型},
  % professional-type* = {Professional degree type},
  % secret-level       = {秘密},     % 绝密|机密|秘密,注释本行则不保密
  % secret-level*      = {Secret},  % Top secret|Highly secret|Secret
  % secret-year        = {10},      % 保密年限
  %
  % 数学字体
  % math-style         = GB,  % 可选:GB, TeX, ISO
  math-font          = xits,  % 可选:stix, xits, libertinus
}


% 加载宏包

% 定理类环境宏包
\usepackage{amsthm}

% 插图
\usepackage{graphicx}

% 三线表
\usepackage{booktabs}

% 跨页表格
\usepackage{longtable}

% 算法
\usepackage[ruled,linesnumbered]{algorithm2e}

% SI 量和单位
\usepackage{siunitx}

% 参考文献使用 BibTeX + natbib 宏包
% 顺序编码制
\usepackage[sort]{natbib}
\bibliographystyle{stuthesis-numerical}

% 著者-出版年制
% \usepackage{natbib}
% \bibliographystyle{stuthesis-authoryear}

% 本科生参考文献的著录格式
% \usepackage[sort]{natbib}
% \bibliographystyle{stuthesis-bachelor}

% 参考文献使用 BibLaTeX 宏包
% \usepackage[style=stuthesis-numeric]{biblatex}
% \usepackage[bibstyle=stuthesis-numeric,citestyle=stuthesis-inline]{biblatex}
% \usepackage[style=stuthesis-authoryear]{biblatex}
% \usepackage[style=stuthesis-bachelor]{biblatex}
% 声明 BibLaTeX 的数据库
% \addbibresource{bib/stu.bib}

% 配置图片的默认目录
\graphicspath{{figures/}}

% 数学命令
\makeatletter
\newcommand\dif{%  % 微分符号
  \mathop{}\!%
  \ifstu@math@style@TeX
    d%
  \else
    \mathrm{d}%
  \fi
}
\makeatother
\newcommand\eu{{\symup{e}}}
\newcommand\iu{{\symup{i}}}

% 用于写文档的命令
\DeclareRobustCommand\cs[1]{\texttt{\char`\\#1}}
\DeclareRobustCommand\pkg{\textsf}
\DeclareRobustCommand\file{\nolinkurl}

% hyperref 宏包在最后调用
\usepackage{hyperref}
